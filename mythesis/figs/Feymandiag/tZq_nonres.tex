\documentclass{standalone}

\usepackage{tikz}
\usepackage{tikz-feynman}
\usepackage{pgfplots}
\usetikzlibrary{positioning,shapes,arrows}
\usetikzlibrary{decorations.pathmorphing}
\usetikzlibrary{decorations.markings}

\begin{document}

\tikzset{
    particle/.style={thick, postaction={decorate},
        decoration={markings, mark=at position 0.6 with {\arrow{triangle 45}}}},
    photon/.style={decorate, draw=black,
        decoration={snake}},
    gluon/.style={decorate, draw=black,
        decoration={coil, segment length=5pt, amplitude=4pt}}
}

\begin{tikzpicture}[node distance=1cm and 2cm]

% Coordinates for quarks and bosons
\coordinate[label=left:$u$] (q1);
\coordinate[right=2cm of q1] (aux1); % u to W vertex
\coordinate[below right=1cm and 0.5cm of aux1] (aux2); % W to l+ vertex
\coordinate[below=1cm of aux2] (aux3); % W to l- vertex
\coordinate[below left=1cm and 0.5cm of aux3] (aux4); % W to b vertex
\coordinate[below left=1.5cm of aux4] (aux5); % g to b vertex

% Gluon and quark labels
\coordinate[left=1.5cm of aux5,label=left:$g$] (g);
\coordinate[right=of aux1,label=right:$d$] (q2);
\coordinate[right=of aux2,label=right:$l^+$] (lplus);
\coordinate[right=of aux3,label=right:$l^-$] (lminus);
\coordinate[right=of aux4,label=right:$t$] (t);
\coordinate[below right=of aux5,label=right:$\bar{b}$] (b);

% Quark lines
\draw[particle] (q1) -- (aux1);
\draw[particle] (aux1) -- (q2);
\draw[particle] (aux4) -- (t);
\draw[particle] (aux5) -- node[label=above left:$b$] {} (aux4);
\draw[particle] (b) -- (aux5);

% Boson lines
\draw[photon] (aux1) -- node[label=right:$W$] {} (aux2);
\draw[photon] (aux4) -- node[label=left:$W$] {} (aux3);
\draw[gluon] (g) -- (aux5);

% Lepton lines
\draw[particle] (lplus) -- (aux2);
\draw[particle] (aux3) -- (lminus);

% Solid line connecting l+ l-
\draw[particle] (aux2) -- (aux3);

\end{tikzpicture}

\end{document}
