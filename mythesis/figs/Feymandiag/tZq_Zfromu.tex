\documentclass{standalone}

\usepackage{tikz}
\usepackage{tikz-3dplot}
\usepackage{pgfplots}
\usetikzlibrary{positioning,shapes,arrows}
\usetikzlibrary{decorations.pathmorphing}
\usetikzlibrary{decorations.markings}

\begin{document}
	\tikzset{
		particle/.style={thick, postaction={decorate},
			decoration={markings, mark=at position 0.6 with {\arrow{triangle 45}}}},
		photon/.style={decorate, draw=black,
			decoration={coil, aspect=0}},
		gluon/.style={decorate, draw=black,
			decoration={coil, segment length=5pt, amplitude=4pt}}
	}


\begin{tikzpicture}[node distance=1cm and 2cm]

\coordinate[label=left:$u$] (q1);
% Definition of all vertices
\coordinate[right=1cm of q1] (aux4); % u u Z
\coordinate[right=1cm of aux4] (aux1); % u d W
\coordinate[below=2cm of aux1] (aux2); % W b t
\coordinate[below left=1.5cm of aux2] (aux3); % g b bbar

% Definition of boson labels
\coordinate[left=1.5cm of aux3,label=left:$g$] (g);
\coordinate[above right=1cm of aux4,label=right:$Z$] (Z);

% Definition of quark labels
\coordinate[right=of aux1,label=right:$d$] (q2);
\coordinate[right=of aux2,label=right:$t$] (t);
\coordinate[below right=of aux3,label=right:$\bar{b}$] (b);

%-------------------------------------------------------------------------

% Quark lines
\draw[particle] (q1) -- (aux4);
\draw[particle] (aux4) -- (aux1);
\draw[particle] (aux1) -- (q2);
\draw[particle] (aux2) -- (t);
\draw[particle] (aux3) -- node[label=above left:$b$] {} (aux2);
\draw[particle] (b) -- (aux3);

% Boson Lines
\draw[photon] (aux1) -- node[label=left:$W$] {} (aux2);
\draw[gluon] (g) -- (aux3);
\draw[photon] (aux4) -- (Z);
\end{tikzpicture}

\end{document}