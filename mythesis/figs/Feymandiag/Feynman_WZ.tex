\documentclass{standalone}

\usepackage{tikz}
\usepackage{tikz-3dplot}
\usepackage{pgfplots}
\usetikzlibrary{positioning,shapes,arrows}
\usetikzlibrary{decorations.pathmorphing}
\usetikzlibrary{decorations.markings}

\begin{document}
	\tikzset{
		particle/.style={thick, postaction={decorate},
			decoration={markings, mark=at position 0.6 with {\arrow{triangle 45}}}},
		photon/.style={decorate, draw=black,
			decoration={coil, aspect=0}},
		gluon/.style={decorate, draw=black,
			decoration={coil, segment length=5pt, amplitude=4pt}}
	}


\begin{tikzpicture}[node distance=1cm and 1.5cm]

\coordinate[label=left:$q$] (q1);
% Definition of all vertices
\coordinate[below right=of q1] (aux1); % qqZ
\coordinate[right=2cm of aux1] (aux2); % Zll
\coordinate[below=2.5cm of aux1] (aux3); % qqg
\coordinate[right=2cm of aux3] (aux4); % gbbar

% Definition of quark labels
\coordinate[below left=of aux3,label=left:$\bar{q}'$] (q2);
\coordinate[below right=1cm of aux4,label=right:$\nu$] (b1);
\coordinate[above right=1cm of aux4,label=right:$l^+$] (b2);

% Definition of lepton labels
\coordinate[above right=1cm of aux2,label=right:$l^+$] (l1);
\coordinate[below right=1cm of aux2,label=right:$l^-$] (l2);

% -------------------------------------------------------------------------

% Draw quark lines

% Draw W boson lines
\draw[photon] (aux1) -- node[label=above:$Z$] {} (aux2);
\draw[photon] (aux3) -- node[label=below:$W^+$] {} (aux4);


% Draw quark lines
\draw[particle] (q1) -- (aux1);
\draw[particle] (aux1) -- (aux3);
\draw[particle] (aux3) -- (q2);

\draw[particle] (aux4) -- (b1);
\draw[particle] (b2) -- (aux4);

% Draw lepton lines
\draw[particle] (aux2) -- (l2);
\draw[particle] (l1) -- (aux2);

\end{tikzpicture}

\end{document}