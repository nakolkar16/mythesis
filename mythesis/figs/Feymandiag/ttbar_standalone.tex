\documentclass{standalone}

\usepackage{tikz}
\usepackage{tikz-3dplot}
\usepackage{pgfplots}
\usetikzlibrary{positioning,shapes,arrows}
\usetikzlibrary{decorations.pathmorphing}
\usetikzlibrary{decorations.markings}

\begin{document}
	\tikzset{
		particle/.style={thick, postaction={decorate},
			decoration={markings, mark=at position 0.6 with {\arrow{triangle 45}}}},
		photon/.style={decorate, draw=black,
			decoration={coil, aspect=0}},
		gluon/.style={decorate, draw=black,
			decoration={coil, segment length=5pt, amplitude=4pt}}
	}


\begin{tikzpicture}[node distance=1cm and 1.5cm]

\coordinate[label=left:$g$] (g1);
% Definition of all vertices
\coordinate[below right=of g1] (aux1); % gtt
\coordinate[right=1cm of aux1] (aux2); % tW+b
\coordinate[above right=1cm of aux2] (aux3); %W+ l+nu
\coordinate[below=2cm of aux1] (aux4); % g tbar bbar
\coordinate[right=1cm of aux4] (aux5); % tbar bbar W-
\coordinate[below right=1cm of aux5] (aux6); %W- l-nu

% Definition of boson labels
\coordinate[below left=of aux4,label=left:$g$] (g2);

% Definition of quark labels
\coordinate[right=of aux2,label=right:$b$] (b1);
\coordinate[right=of aux5,label=right:$\bar{b}$] (b2);

% Definition of lepton labels
\coordinate[right=1cm of aux3,label=right:$l^+$] (l1);
\coordinate[above right=1cm of aux3,label=right:$\nu$] (nu1);

\coordinate[right=1cm of aux6,label=right:$l^-$] (l2);
\coordinate[below right=1cm of aux6,label=right:$\bar{\nu}$] (nu2);

% -------------------------------------------------------------------------

% Draw quark lines
\draw[gluon] (g1)   -- (aux1);
\draw[gluon] (g2)   -- (aux4);

% Draw W boson lines
\draw[photon] (aux2) -- node[label=above:$W^+$] {} (aux3);
\draw[photon] (aux5) -- node[label=below:$W^-$] {} (aux6);

% Draw quark lines
\draw[particle] (aux1) -- node[label=above:$t$] {} (aux2);
\draw[particle] (aux4) -- (aux1);
\draw[particle] (aux5) -- node[label=below:$\bar{t}$] {} (aux4);
\draw[particle] (b2) -- (aux5);
\draw[particle] (aux2) -- (b1);

% Draw lepton lines
\draw[particle] (aux3) -- (nu1);
\draw[particle] (l1) -- (aux3);

\draw[particle] (aux6) -- (l2);
\draw[particle] (nu2) -- (aux6);

\end{tikzpicture}

\end{document}