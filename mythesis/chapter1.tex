% !TEX root = mythesis.tex

%==============================================================================
\chapter{Theoretical Concepts and Experimental Basics}
\label{sec:SM}
%==============================================================================

Since many years, physical phenomena occuring around us has shaped our understanding about nature.
The Standard Model (SM) of particle physics is a theory that explains almost everything
that nature has to offer. It is based on fundamental particles and their interactions
being governed by Quantum Field Theories (QFTs). 

The Standard Model is divided into spin-1 fermions and spin-0 bosons. The fermions are
further divided into leptons and quarks as shown in \mynote{FIGURE}{Add figure}. Another
classification of fermions is into generations. The first generation consstitutes
\Pup, \Pdown, \Pelectron and \Pnue which give rise to matter around us. The second
and third generation particles are high energy \textit{siblings} of the first generation
particles. These are observed at high energies such as colliders. The SM also considers
anti-particles which are clones of particles with opposite quantum numbers.

These fermions interact with each other via exchanging bosons which are also called 
\textit{force-carrier} particles. The massless neutral photon (\Pphoton) is the 
messenger of the electromagnetic (EM) force which is experienced by charged particles.
The underlying QFT is called Quantum Electrodynamices (QED). The electrostatic 
attraction between charged particles is the low-energy manifestation of QED. Among 
the SM, all fermions except neutrinos are sensitive to the EM force. The underlying symmetry
is the U(1) symmetry. 

The strong interaction, mediated by massless gluons, is experienced by particles 
carrying the so-called colour charge. The physics behind the strong interaction is 
explained in Quantum Chromodynamics (QCD). Only the quarks can interact via the 
strong interaction. A peculiar thing in QCD is that the gluons themselves also carry 
colour charge. 

The weak force carriers are \PWpm and \PZ, which unlike \Pphoton and 
gluons, are massive and charged in case of \PWpm. The weak interaction manifests
itself in phenomena such as, $\beta$-decay and fusion processes inside the sun. All the
SM particles, including the neutrinoes, can feel the weak force. 
Before discussing the elecroweak unification, let's dive into the weak interaction. 
The interaction mediated by \PWpm and \PZ is called charged-current weak interaction and
neutral-current weak interaction, respectively. The famous Wu experiment proved that the charged
current weak interaction violates parity. The parity violating nature of the weak interaction
suggests that the interaction vertex must be different from that of QED and QCD. The weak 
interaction is described using a $V-A$ vertex and this fact imples that only left-handed
chiral particle states and right-handed chiral antiparticle states can participate in 
charged-current weak interaction. 

The last piece of the SM puzzle is the Higgs boson which is a spin-0 boson. All the
particles gain their mass by Higgs mechanism. 

\subsection{Feynman diagrams}

Interactions between the SM particles can give rise to various processes. In order to
visualise it, a tool called Feynman diagrams is widely used. These diagrams are 
pictorial representations of the interactions which makes use of straight lines with 
arrows to show particles and anti-particles. Moreover, curly lines are used to show
the boson exchanged between them. The Feynman diagrams are symbolic and have no 
physical meaning. \mynote[inline]{}{Add and explain an example feynman diagram}
\mynote[inline]{}{Explain matrix element, cross section and differential cross-section calculation}

\section{The Strong Force}
Electrons and nucleus inside an atom are held together by the electromagnetic force. The same 
force also exists between protons inside the nucleus causing repulsion. However, there exists a force
which is strong enough to overcome repulsion and keep the nucleus together. It is called the strong force or 
the strong nuclear force. The QFT describing the strong force is called Quantum Chromodynamics (QCD) and the
underlying symmetry group is SU(3) described by $3 \times 3$ matrices. The eight generators of the SU(3) group
give rise to eight gluons which are the strong force mediators. The structure of the SU(3) group demands that 
the wave function of the strongely interacting particle must be a 3-component vector. This gives rise to 
a new degree of freedom called "colour", with three states called red,blue and green. Consequently, particles
having a non-zero colour charge can feel the strong force. Among the SM particles, only quarks have the colour
charge which can be either red, blue or green.

A major differentiating factor between QCD and QED is that the gauge boson in QCD carries the charge of 
interaction. In other words, gluons also carry the colour charge which allows them to interact with other 
gluons as well. As a result of this self-interaction, no coloured object can be found as a free particle in 
nature. Due to this so-called colour confinement, quarks cannot exist independently but instead are found in
colour-neutral states called \textit{hadrons}. For instance, if two quarks are pulled away from each other, a gluon field is 
created between them which is proportional to the separation. The gluon fields is so strong that at some point,
the energy in this field is sufficient to produce new quarks and antiquarks that form colourless bound states.
This process is called hadronisation. Due to colour confinement, only certain configurations for hadrons
are permitted. The possible combinations discovered so far can be categorised into mesons (\Pquark\APquark),
baryons (\Pquark\Pquark\Pquark) and antibaryons (\APquark\APquark\APquark).


\section{The Electroweak theory}
In the 1960s, physicists were trying to formulate a gauge theory for weak interactions
similar to QED. A theory can be a gauge theory if it has an underlying mathematical symmetry
and it is renormalisable\footnote{A quantum field theory is renormalisable if...}. Glashow,
Salam and Weinberg discovered such a gauge theory by unifying electromagnetic force and 
the weak force.

The electroweak (EW) theory is a unification of QED and the thoery of weak interactions. 
It is described by the symmetry group $SU(2)_L \otimes U(1)_Y$. The corresponding
charges of the electroweak theory are the weak isospin $I,I_3$ and the weak hypercharge $Y$.
The weak hypercharge $Y$ determines the interaction under the $U(1)$ transformations.
The weak isospin of particles determines their transformation under $SU(2)$ and therefore, it is
used to make multiplets of particles. The left-handed leptons ($\Plepton_L$) will form doublets
as shown in \mynote{EQUATION}{fix the format and label reference} because they transform into each other under the influence of weak force. 
This is due to to the $V-A$ vertex form of the weak interaction. On the other hand, the right-handed 
particles are singlets($\Plepton_R$). 

\begin{align*}
    \Plepton_R = \Pelectron_R, \Pmuon_R, \Ptau_R \\
    \Plepton_L = \begin{pmatrix} \Pnue \\ \Pelectron \end{pmatrix}_L , \begin{pmatrix} \Pnum \\ \Pmuon \end{pmatrix}_L , \begin{pmatrix} \Pnut \\ \Ptau \end{pmatrix}_L 
\end{align*}

The Lagrangian of the EW model introduces three bosons $\PW_\mu^{(1,2,3)}$ corresponding
to $SU(2)$ and one $B_\mu$ corresponding to $U(1)$. Th experimentally observed $\PWpm$
are combination of $\PW_\mu^{(1)}$ and $\PW_\mu^{(2)}$ whereas photon ($A$) and the Z-boson are linear
combinations of $\PW_\mu^{(3)}$ and $B_\mu$ based on the weak mixing angle ($\theta_W$)
as given below:

\begin{align*}
A_\mu = +B_\mu \text{cos} \theta_W + W_\mu^{(3)}\text{sin} \theta_W \\
Z_\mu = -B_\mu \text{sin} \theta_W + W_\mu^{(3)}\text{cos} \theta_W \\
\end{align*}

The weak interaction for the quark sector can be explained by creating similar
$SU(2)$ doublets(Q).

\begin{align*}
    Q = \begin{pmatrix} u \\ d' \end{pmatrix}, \begin{pmatrix} c \\ s' \end{pmatrix}, \begin{pmatrix} t \\ b' \end{pmatrix}
\end{align*}

The strength of the weak interactions for quarks is determined experimentally by studying
nuclear $\beta$-decay. It is observed that the vertices corresponding to different quark
flavours have different coupling strengths. The reason for this is given by the Cabibo
hypothesis which states that, the flavour eigen states that participate in the weak interactions
are a mixture of the mass eigen states. The relation between them is given by the 
Cabibo-Kobayashi-Maskawa (CKM) matrix. 

\begin{align*}
    \begin{pmatrix} d' \\ s' \\ b'\end{pmatrix}
     = \begin{pmatrix} V_{ud} & V_{us} & V_{ub} \\
                       V_{cd} & V_{cs} & V_{cb} \\
                       V_{td} & V_{ts} & V_{tb}
    \end{pmatrix} \begin{pmatrix} d \\ s \\ b\end{pmatrix}
\end{align*}

The values of the CKM matrix elements can be found in \cite{pdg2024}. The diagonal
of the matrix is close to unity, suggesting that the weak interaction is stronger within
the same generation of quarks. 

The experiments at the Gargamelle bubble chamber in 1973 hinted the evidence of a neutral
massive boson responsible for the observed neutrino interactions~\cite{HASERT1973138}. In 1983, the \PZ-boson
was directly discovered at the Super-Proton Synchrotron at CERN. The electroweak theory
was verified by this pathbreaking discovery. The properties of the \PZ-boson were
studied at the Large Electron-Positron (LEP) collider at CERN. The discovery of \PZ and \PW 
bosons are among the crucial tests of the Standard Model. 


\section{The Higgs mechanism}

\section{Physics at the hadron colliders}

\subsection*{Parton Distribution Functions(PDFs)}
Protons at the LHC collide at high energies giving rise to deep inelastic interactions called hard processes.
In such cases, the interactions are not between protons but between their constituents which are
quarks and gluons, collectively known as \textit{partons}. These partons carry a fraction of the total
momentum of the proton (or a hadron in general). In order to study an interaction, it is important to
know the effective energy of the interacting partons and their flavour. This information is encoded in 
the Parton Distribution Functions (PDFs). It describes the probability of finding a parton of 
certain flavour $i$, carrying a momentum fraction $x_i$ at a certain energy scale.

\subsection*{Pileup}
The 
primary hard scatter collisions, that are of interest, are contaminated by soft interactions 
called pileup. It is defined by the average number of interactions
recorded per bunch crossing. Sources of pileup are categorized into in-time and out-of-time pileup. In-time pile up is due to collisions
occurring in the same bunch-crossing and out-of-time pile-up is contributed by the collisions from previous
or later bunches. Some of the sub-detectors have sensitivity windows longer than the interval between
bunch crossings. This eventually affects the recorded number of interactions per bunch. The accurate
detection of objects under study becomes difficult due to pile-up events. The higher the 
luminosity, more the pileup. The object reconstruction algorithms have dedicated procedures 
to mitigate pileup.

\subsection*{Luminosity and cross-section ($\sigma$)}
The quantity that measures the ability of a collider to produce particle interactions is called
instantaneous luminosity ($\mathcal{L}$). The instantaneous luminosity integrated over the lifetime
of collider operation is called integrated luminosity ($L$).

In order to define the event rate for interesting processes, along with luminosity, we require
another quantity called the cross-section. At the subatomic scale, the particle interactions 
are governed by laws of quantum physics. Therefore, a theory can predict 
the \textit{probablility} of certain outcomes of collisions. The probablity that a certain
process will take place is called its cross-section ($\sigma$). Finally, the number of event rate of
specific interactions is defined as the product of integrated luminosity and the cross-section (\cref{eq:lumi}).
\begin{equation}
    R = \sigma \cdot \int_{dt} \mathcal{L}(t)
    \label{eq:lumi}
\end{equation}

For a particle collider, beam energies and the luminosity are two important figures of merit. 
High energy allows the production of new heavy particles and high luminosity allows more flux 
of particles contributing to high number of collisions.


\mynote[inline]{}{Lepton universality..because tZq trilepton, decay into muons and
electrons equal probability}

\mynote[inline]{}{branching ratio}


\section{Top quark physics}
The top quark was discovered in 1995 at the Tevatron laboratory(?) by the CDF collaboration[CITE].
It is the heaviest fundamental particle discovered so far with a mass of \SI{173}{\GeV}[CITE PDG] 
and since it is close to the EW scale, the top quark might play an important role in understanding
Electroweak Symmetry Breaking (EWSB). Moreover, the fact that $m_t$ is greater than $m_H$ hints whether
the top quark gets its mass through Higgs mechanism.  

The top quark is the weak isospin partner of the bottom quark and thus completes the three 
generation structure of the SM. Since its discovery, top quark has been a crucial part of
any physics program at the hadron colliders because of its heavy mass. It decays almost exclusively
into W and b before hadronisation. This property gives us a unique opportunity to study a "bare"
quark because some top quark properties are conserved in the decay process and passed on to 
its decay products.

The LHC is a top factory becuase it can produce abundant top quarks and other related processes.
Therefore, LHC provides a lot of data which is analysed extensively to understand the top quark and
its properties. The production of top quarks is mainly through three kinds of processes: Wt-channel,
t-channel and the s-channel.









