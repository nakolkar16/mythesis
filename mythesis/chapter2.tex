% !TEX root = mythesis.tex

%==============================================================================
\chapter{Experimental equipment}
\label{sec:LHCATLAS}
%==============================================================================

\section{The Large Hadron Collider}
The Large Hadron Collider is a two-ring proton-proton collider situated near Geneva across the Swiss-French border. It is built in the 26.7 km long tunnel 
that previously housed the Large Electron-Positron (LEP) collider. The two proton beams are accelerated in opposite directions and brought into collision at 
four points where the four detectors are located. The LHC has two tranfer tunnels of 2.5 km each that connects to the CERN accelerator complex as shown in 
Figure (ADD FIGURE REF). Each system in this complex contributes to the acceleration of protons. Electrons are stripped off from hydrogen atoms and the remaining hydrogen nuclei, nothing but protons
enter the LINAC2 which is the first system in the CERN accelerator complex. Then the accelerated protons are fed into the Proton Synchrotron Booster (PBS)
and eventually into Super Proton Synchrotron (SPS). At this point proton energy is 450 GeV when they are finally injected into two opposing beams of the
LHC. Here each beam attains a path-breaking energy of 6.5 TeV. 

The proton beams need to be directed along the circular structure of the LHC and to achieve that superconducting magnets are placed around the beam pipe.
Due to the limited area in the LEP tunnel, a single magnet system is shared by both the beam pipes. The superconducting coils are immersed in a superfluid Helium bath that is cooled down to a certain temperature to achieve superconductivity. One of the 
challenges here is synchrotron radiation. The accelerating protons radiate photons when they are accelerated and this radiation can adversely impact the 
temperature inside the cryogenic (?). To fix this problem there are beam screens placed between the beam pipe and the magnet system that reflects or absorbs 
these photons. 



Fixed target and collision with other particles, th elatter gives the hightest CM energies(refer teracslae book)

The LHC has two tranfer tunnels of 2.5 km each that connects to the CERN accelerator complex. 

Macimum energy and achievable luminosity are important design parameters of an accelerator. High energy allows production of new heavy particles and high 
lumi allows more flux of particles contributing to high number of collisions.