% !TEX root = mythesis.tex

%==============================================================================
\chapter{Introduction}
\label{sec:intro}
%==============================================================================

It is always a pleasure for a student to get an opportunity to converse with a scientist. When this opportunity presented itself, I began describing my 
interest in scientific research, especially Astronomy. The scientist gazed at me and asked how could I be sure of pursuing Astronomy at such an early
stage of my career. This is when I started wondering what else is there for me to explore. I was opportunate to work under the same scientist and explore the
fundamental world. The different fundamental particles were like a lego set for me and their properties was like building a lego structure of nature itself!
The Standard Model (SM) of particle physics is a theory capable of explaining three of the four fundamental forces of nature (rephrase). 

It is no surprise that the most famous particle in the SM is the Higgs Boson. I was intriguied by what this particle has to offer and why is it so important.
My exploration of particles during my Masters led me to get to know the top-quark. Being the heaviest fundamental particle discovered so far, it sure had a lot 
to offer. I was specifically attracted to the fact that top-quark is heavier than even the Higgs boson. Working in the Brock reserach group, I leant a lot about
single \Ptop-quark and its interactions. I not only increased my knowledge but also had a first hand experience of research at CERN. 

This thesis focuses on the single \Ptop-quark and the \PZ-boson. It presents the differential cross-section measurement of the $tZq$ production using 
the Run-2 data of the ATLAS detector at CERN. The \Ptop-quark and \PZ-boson decay and among the various decay mode combinations, this thesis focuses on the 
the decay mode where both the particles decay into leptons. This is the so-called trilepton decay channel. It is a clean decay mode compared to other modes
due to the fact that reconstructing leptons is less of a mess than reconstructing quarks. Moreover, this channel is statistically limited which presents a 
challenge when measuring the differential cross-section. The measurement is performed using the method of unfolding. 

In general, any apparatus used for measurement brings with it an extent of reliability on its results. In other words, no apparatus is ideal. The collision
data recorded by the detector is distorted due to certain effects and these distortions are undesirable for analysis. Unfolding is a mathematical tool to 
correct for detector distortions. This thesis presents the implementation of Profile-likelihood unfolding (PLU) to measure the differential cross-section. 
Various tests were performed to test the robustness of the method which are also presented. 

new Particle physics motivation: Energy frontier and precision frontier. Energy E=mc2. more energy more chance to create particles. and precsion heisenberg proinviple.

(Write about what chapters present what.)