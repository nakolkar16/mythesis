% !TeX root = thesis_guide.tex
% chktex-file 1 chktex-file 46

%==============================================================================
\chapter{Making a glossary or list of acronyms}%
\label{sec:app:glossary}
%==============================================================================

\LaTeX\ file: \url{run:./guide_appendix_glossary.tex}{guide\_appendix\_glossary.tex}\\[1ex]
\noindent
What is the difference between a glossary and a list of acronyms? They
are in fact very similar. A list of acronyms is probably most
appropriate for defining the abbreviations used for detector names,
e.g.\ the electromagnetic calorimeter (EMC). A glossary is used for
defining terms, e.g.\ \enquote{\textbf{transition radiation}: transition
radiation can be emitted when a particle crosses the boundary between
two media with different dielectric constants}. A list of acronyms is
probably a good idea to include in any thesis; a glossary may be
helpful as well.

Several packages are available to help in the creation of a glossary:
\begin{description}
\item[\Package{glossaries}] A new glossary package that can do everything!
  This is what I use here.
\item[\Package{nomencl}] Another glossary package that I used in the book
  \enquote{Physics at the Terascale}.
\item[\Package{glosstex}] Rather an old package that looks to be quite
  simple to use.
\item[\Package{acronym}] A fairly extensive package for acronyms.
\end{description}

A short introduction to the first three packages can be found in the \TeX nische
Komödie 4/2012.
\Package{acronym} is a further package that provides quite a few commands
associated with acronyms.
It is probably sufficient if you only want to define acronyms and provide a list of them.
I have not actually tried out the package,
but as far as I can tell you should provide a definition of all acronyms
at the place you want to output the list within an \Env{acronym} environment.

The \Package{nomencl} package is maybe the easiest to use for creating
a simple list of acronyms. If you want to change the formatting or do
anything other than create a simple list, it is probably worth
investigating \Package{glossaries}.

The nice thing about \Package{glossaries} is that it works with
\Package{hyperref} so that you can even click on an acronym and find
its definition. In this guide I included the package with the option
\Option{acronym}, in order to get both a glossary and a list of
acronyms. I also included the option \Option{toc}, so that the
glossary and list of acronyms are included in the table of
contents. You tell it to to make a glossary (and list of acronyms) by
giving the command \Macro{makeglossaries} in the preamble and you
print the glossary (usually at the end of the document) with the
\Macro{printglossaries} command. I have not included these commands in
the thesis skeleton, as the \Package{glossaries} package is rather new
and may not be available in older \TeX\ installations. In order to
process the glossary, you need to run the command
\texttt{makeglossaries filename}. This is included in the
\texttt{Makefile} for the commands \texttt{make guide}
and should be added to \texttt{make thesis}
if you want to make a glossary for your thesis.

If you use the package \Package{xtab} for long tables,
then you should include \Package{glossaries} with the option \Option{nosuper}
to avoid it loading \Package{supertabular}.
Use styles with a \texttt{long} rather than a \texttt{super} variant if you want such features.

If you have a glossary and want it to compile it using \Command{latexmk}\index{latexmk}
you should add the following
lines to \File{\textasciitilde/.latexmkrc} or to \File{\textasciitilde/.config/latexmk/latexmkrc}:\footnote{%
I got this information from \url{https://tex.stackexchange.com/questions/1226/how-to-make-latexmk-use-makeglossaries}}
\begin{bashlisting}
add_cus_dep( 'acn', 'acr', 0, 'makeglossaries' );
add_cus_dep( 'glo', 'gls', 0, 'makeglossaries' );
$clean_ext .= " acr acn alg glo gls glg";
sub makeglossaries {
  my ($base_name, $path) = fileparse( $_[0] );
  pushd $path;
  my $return = system "makeglossaries", $base_name;
  popd;
  return $return;
}
\end{bashlisting}
\noindent This code can be found in the file \File{latexmk/latexmkrc}.

I illustrate the use with the standard description of the ZEUS
detector, which includes the abbreviations that are used in the rest
of a ZEUS paper.  As a first step you define the terms for the
glossary and the acronyms.  In this guide, the acronyms and the
glossary entries are defined in the file
\url{guide_glossary.tex}. These definitions must come before
they are used. You can decide whether it is best to have them in a
single file, or define them just before they are used. When they are
printed at the end, they will be sorted alphabetically.

Once an acronym is defined, the first time the acronym is used,
\Macro{gls}\{acronym\}, the full text and the abbreviation is
printed. Every time after that only the abbreviation is printed.

\section{ZEUS detector description}%
\label{sec:app:glossary:zeus}

\begin{tcblisting}{}
In the kinematic range of the analysis, charged particles were tracked
in the \gls{CTD} and the \gls{MVD}.
These components operated in a magnetic field of \qty{1.43}{\tesla}
provided by a thin superconducting solenoid.
The \gls{CTD} consisted of 72~cylindrical drift-chamber layers,
organised in nine superlayers covering the polar-angle region
$\ang{15} < \theta < \ang{164}$.
%
The \gls{MVD} consisted of a barrel (BMVD) and a forward (FMVD)
section. The BMVD contained three layers and provided polar-angle
coverage for tracks from \ang{30} to \ang{150}. The four-layer FMVD
extended the polar-angle coverage in the forward region to
\ang{7}. After alignment, the single-hit resolution of the MVD was
\qty{24}{\micro\metre}. The \gls{DCA} of tracks to the nominal vertex in
$X$--$Y$ was measured to have a resolution, averaged over the
azimuthal angle, of $(46 \oplus 122 / p_{T})\,\unit{\micro\metre}$,
with $p_{T}$ in \unit{\GeV}.
For \gls{CTD}-\gls{MVD} tracks
that pass through all nine \gls{CTD} superlayers, the momentum
resolution was $\sigma(p_{T})/p_{T} = 0.0029 p_{T} \oplus 0.0081
\oplus 0.0012/p_{T}$, with $p_{T}$ in \unit{\GeV}.
\end{tcblisting}

\begin{tcblisting}{}
The high-resolution \gls{CAL} consisted of three parts: the forward
(FCAL), the barrel (BCAL) and the rear (RCAL) calorimeters. Each part
was subdivided transversely into towers and longitudinally into one
electromagnetic section (EMC) and either one (in RCAL) or two (in BCAL
and FCAL) hadronic sections (HAC). The smallest subdivision of the
calorimeter was called a cell.  The \gls{CAL} energy resolutions, as
measured under test-beam conditions, were $\sigma(E)/E=0.18/\sqrt{E}$
for electrons and $\sigma(E)/E=0.35/\sqrt{E}$ for hadrons, with $E$ in
\unit{\GeV}. 

ZEUS had quite a large assortment of tracking detectors. In the
forward direction there was the \gls{FTD} and the \gls{TRD}. The
\gls{TRD} was replaced by the \gls{STT} for the HERA~2 running
period.  In the middle of the detector, the \gls{CTD} was always
there, while the \gls{MVD} was also installed later.
\end{tcblisting}
