\chapter{Jet processes}
%\label{sec:MJ}

This section describes the MC samples used for the modelling of multijet production.
\Cref{subsec:jets-pythia} describes the \PYTHIA[8] samples,
\cref{subsec:jets-herwig} describes the \HERWIG[7] samples,
\cref{subsec:jets-powheg} describes the \POWPY[8] samples,
and finally \cref{subsec:jets-sherpa} describes the \SHERPA samples.

\section[Pythia 8]{\PYTHIA[8]}
\label{subsec:jets-pythia}

The descriptions below correspond to the samples in \cref{tab:mj_pythia}.

\begin{table}[!htbp]
  \caption{Nominal multijet samples produced with \PYTHIA.}%
  \label{tab:mj_pythia}
  \centering
  \begin{tabular}{l l}
    \toprule
    DSID range & Description \\
    \midrule
    364700--364712 & \PYTHIA with shower weights \\
    \bottomrule
  \end{tabular}
\end{table}

\paragraph{Description:}

Multijet production was generated using \PYTHIA[8.230]~\cite{Sjostrand:2014zea} with leading-order matrix elements
for dijet production which were matched to the parton shower.
%  supplemented by matrix element corrections (MEC) for the hardest
%  emission.

The renormalisation and factorisation scales were set to the geometric
mean of the squared transverse masses of the two outgoing particles in the matrix element,
\(\pTHatPythia = \sqrt{(\pTX[2][1] + m_1^2) (\pTX[2][2] + m_2^2)}\).
The \NNPDF[2.3lo] PDF set~\cite{Ball:2012cx} was used in
the ME generation, the parton shower, and the simulation of the
multi-parton interactions. The A14~\cite{ATL-PHYS-PUB-2014-021}
set of tuned parameters was used. Perturbative uncertainties were estimated
through event weights~\cite{Mrenna:2016sih} that encompass variations
of the scales at which the strong coupling constant is evaluated in
the initial- and final-state shower as well as the PDF uncertainty in
the shower and the non-singular part of the splitting functions.


\paragraph{Additional description:}

The modelling of fragmentation and
hadronisation was based on the Lund string
model~\cite{Andersson:1983ia,Sjostrand:1984ic}. To populate the
inclusive jet \pT spectrum efficiently, the sample used a biased
phase-space sampling which was compensated for by a continuously decreasing
weight for the event. Specifically, events at a scale
\pTHatPythia scale were oversampled by a factor of
\((\pTHatPythia/\qty{10}{\GeV})^4\).


\section[Herwig 7.1]{\HERWIG[7.1]}
\label{subsec:jets-herwig}

The descriptions below correspond to the samples in \cref{tab:mj_herwig}.

\begin{table}[!htbp]
  \caption{Multijet samples produced with \HERWIG[7].}%
  \label{tab:mj_herwig}
  \centering
  \begin{tabular}{l l}
    \toprule
    DSID range & Description \\
    \midrule
    364922--364929 & angular ordering in shower \HERWIG[7] \\
    364902--364909 & dipole shower \HERWIG[7] \\
    \bottomrule
  \end{tabular}
\end{table}

\paragraph{Description:}

Multijet production at next-to-leading order (NLO) was generated using \HERWIG[7.1.3]~\cite{Bellm:2017jjp}.
The renormalisation and factorisation scales were set to the \pT of the leading jet. The
\MMHT[nlo]~\cite{Harland-Lang:2014zoa} PDF set was used for the matrix element calculation.
Two sets of samples were generated, where one makes use of the default parton shower with angular ordering,
and the other uses the dipole shower as an alternative. The description of
hadronisation was based on the cluster model~\cite{Winter:2003tt} for both of these samples.
Two different samples with the same matrix elements and hadronisation allow the effects of using different
parton shower models to be investigated. These samples include variations from the hard scattering and shower.


\section[Powheg+Pythia8]{\POWPY[8]}
\label{subsec:jets-powheg}

The descriptions below correspond to the samples in \cref{tab:mj_powheg}.

\begin{table}[!htbp]
  \caption{Multijet samples produced with \POWHEGBOX[v2].}%
  \label{tab:mj_powheg}
  \centering
  \begin{tabular}{l l}
    \toprule
    DSID range & Description \\
    \midrule
    361281--361289 & \POWPY[8] \\
    \bottomrule
  \end{tabular}
\end{table}

\paragraph{Description:}

Alternative samples of multijet production at NLO accuracy were produced with \POWHEGBOX[v2]~\cite{Nason:2004rx, Frixione:2007vw}
interfaced to \PYTHIA[8]. These were generated with the dijet process as implemented in \POWHEGBOX[v2]~\cite{Alioli:2010xd}.
The \pT of the underlying Born configuration was taken as the renormalisation and factorisation scales
and the \NNPDF[3.0nlo]~\cite{Ball:2014uwa} parton distribution function (PDF) was used. \PYTHIA with the A14 tune and the
\NNPDF[2.3lo]~\cite{Ball:2012cx} PDF was used for the shower and multi-parton interactions.
These samples included per-event weight variations for different perturbative scales in the matrix element,
different parton distribution functions and their uncertainties, and the \PYTHIA perturbative
shower uncertainties.


\section[Sherpa 2.2]{\SHERPA[2.2]}
\label{subsec:jets-sherpa}

The descriptions below correspond to the samples in \cref{tab:mj_sherpa}.

\begin{table}[!htbp]
  \caption{Multijet samples produced with \SHERPA.}%
  \label{tab:mj_sherpa}
  \centering
  \begin{tabular}{l l}
    \toprule
    DSID range & Description \\
    \midrule
    364677--364685 & \SHERPA AHADIC \\
    364686--364694 & \SHERPA Lund \\
    \bottomrule
  \end{tabular}
\end{table}

\paragraph{Description:}

Multijet production samples were also generated using the \SHERPA[2.2.5]~\cite{Bothmann:2019yzt} generator.
The matrix element calculation was included for the \(2\rightarrow2\) process at leading order, and the default \SHERPA parton
shower~\cite{Schumann:2007mg} based on Catani--Seymour dipole factorisation was used for the showering with \pT
ordering, using the \CT[14nnlo] PDF set~\cite{Dulat:2015mca}.
The first of these samples made use of the dedicated \SHERPA AHADIC model for hadronisation~\cite{Winter:2003tt},
based on cluster fragmentation ideas. A second sample was generated with the same configuration but using the \SHERPA interface to
the Lund string fragmentation model of \PYTHIA[6]~\cite{Sjostrand:2006za} and its decay tables.
These two sets of samples were used to evaluate uncertainties stemming from the hadronisation modelling.

% need to comment on the slicing of these samples. JZ1 to JZ4 are JZW filtered to improve statistics for and an equal number of simulated events, while slices JZ9-JZ12 have been combined into a single JZ9 plus slice.
