% !TEX root = MC_snippets.tex

%%%%%%%%%%%%%%%%%%%%%%%%%%%%%%%%%%%%%%%%%%%
%%%              tW                     %%%
%%%%%%%%%%%%%%%%%%%%%%%%%%%%%%%%%%%%%%%%%%%
\section{Single-top \texorpdfstring{\(tW\)}{tW} associated production}
\label{subsec:tW}

This section describes the MC samples used for the modelling of single-top \(tW\) associated production.
\Cref{subsubsec:tW_PP8} describes the \POWPY[8] samples -- both for the diagram removal (DR) set-ups,
which are used for the nominal prediction as well as uncertainties due to additional radiation and PDFs,
and for the diagram subtraction (DS) set-ups, which are used for the uncertainty due to the treatment
of the overlap with \ttbar production.
\Cref{subsubsec:tW_PH7} describes the \POWHER[7] samples used for the uncertainty
due to parton showering and hadronisation modelling, and \cref{subsubsec:tW_aMCP8} describes the
\MGNLOPY[8] samples used for the uncertainty due to the choice of matching scheme.

The reference cross-section values are extracted from Ref.~\cite{LHCTopWGsgtopXsec}.

\subsection[Powheg+Pythia8]{\POWPY[8]}
\label{subsubsec:tW_PP8}

\paragraph{Samples}
%\label{par:tW_PP8_samples}

\Cref{tab:tW_PP8} gives the DSIDs of the \(tW\) \POWPY[8] samples, for both the DR and DS schemes.
Single-top and single-anti-top (\(tW^-\) and \(\bar{t}W^+\)) events were generated in different samples.
The dileptonic samples overlap with the inclusive ones.

\begin{table}[htbp]
  \caption{Single-top \(tW\) associated production samples produced with \POWPY[8].}%
  \label{tab:tW_PP8}
  \centering
  \begin{tabular}{l l}
    \toprule
    DSID & Description \\
    \midrule
    410646 & \(tW^-\) (DR) inclusive \\
    410647 & \(\bar{t}W^+\) (DR) inclusive \\
    410648 & \(tW^-\) (DR) dileptonic \\
    410649 & \(\bar{t}W^+\) (DR) dileptonic \\
    \midrule
    410654 & \(tW^-\) (DS) inclusive \\
    410655 & \(\bar{t}W^+\) (DS) inclusive \\
    410656 & \(tW^-\) (DS) dileptonic \\
    410657 & \(\bar{t}W^+\) (DS) dileptonic \\
    \bottomrule
  \end{tabular}
\end{table}

\paragraph{Short description:}

The associated production of top quarks with \(W\) bosons (\(tW\)) was
modelled by the
\POWHEGBOX[v2]~\cite{Re:2010bp,Nason:2004rx,Frixione:2007vw,Alioli:2010xd}
generator at NLO in QCD using the five-flavour scheme and the
\NNPDF[3.0nlo] set of PDFs~\cite{Ball:2014uwa}.
The diagram removal scheme~\cite{Frixione:2008yi} was used to
remove interference and overlap with \ttbar production.
The related uncertainty was estimated by comparison with an alternative sample
generated using the diagram subtraction scheme~\cite{Frixione:2008yi,ATL-PHYS-PUB-2016-020}.\footnote{Analyses which do not use this approach
should obviously not use this sentence in their description.}
The events were interfaced to \PYTHIA[8.230]~\cite{Sjostrand:2014zea} using the A14
tune~\cite{ATL-PHYS-PUB-2014-021} and the \NNPDF[2.3lo] set of
PDFs~\cite{Ball:2012cx}.

The uncertainty due to initial-state radiation (ISR) was estimated by
simultaneously varying the \hdamp parameter and the \muR and
\muF scales, and choosing the Var3c up/down variants of the A14 tune
as described in Ref.~\cite{ATL-PHYS-PUB-2017-007}. The impact of
final-state radiation (FSR) was evaluated by varying the renormalisation scale
for emissions from the parton shower up or down by a factor two.



\paragraph{Long description:}

Single-top \(tW\) associated production was modelled using the
\POWHEGBOX[v2]~\cite{Re:2010bp,Nason:2004rx,Frixione:2007vw,Alioli:2010xd}
generator, which provided matrix elements at next-to-leading
order~(NLO) in the strong coupling constant \alphas\ in the five-flavour
scheme with the \NNPDF[3.0nlo]~\cite{Ball:2014uwa} parton
distribution function~(PDF) set.  The functional form of the
renormalisation and factorisation scales was set to the default scale,
which is equal to the top-quark mass (\(\mtop = \qty{172.5}{\GeV}\)).
The diagram removal scheme~\cite{Frixione:2008yi} was employed to handle the interference
with \ttbar production~\cite{ATL-PHYS-PUB-2016-020}.  The events were
interfaced with \PYTHIA[8.230]~\cite{Sjostrand:2014zea} using the A14
tune~\cite{ATL-PHYS-PUB-2014-021} and the \NNPDF[2.3lo] PDF set.  The
decays of bottom and charm hadrons were simulated using the
\EVTGEN[1.6.0] program~\cite{Lange:2001uf}.

The inclusive cross-section was corrected to the theory prediction
calculated at NLO in QCD with NNLL soft-gluon
corrections~\cite{Kidonakis:2010ux,Kidonakis:2013zqa}.  For proton--proton
collisions at a centre-of-mass energy of \(\rts = \qty{13}{\TeV}\), this
cross-section corresponds to \(\sigma(tW)_{\text{NLO+NNLL}} = 71.7 \pm 3.8\,\unit{\pb}\),
using a top-quark mass of \(\mtop = \qty{172.5}{\GeV}\).  The uncertainty in
the cross-section due to the PDF was calculated using the \MSTW[nnlo] 90\%
CL~\cite{Martin:2009iq,Martin:2009bu} PDF set, and was added in
quadrature to the effect of the scale uncertainty.

The uncertainty due to initial-state radiation (ISR) was estimated by
comparing the nominal \ttbar sample with two additional
samples~\cite{ATL-PHYS-PUB-2017-007}.  To simulate higher parton
radiation, the factorisation and renormalisation scales were reduced by
a factor of 0.5 while simultaneously increasing the \hdamp value to
3.0\,\mtop and using the Var3c up variation from the A14 tune. For
lower parton radiation, \muR and \muF were increased by a factor of two
while keeping the \hdamp value set to 1.5\,\mtop and using the Var3c down
variation in the parton shower.  The Var3c A14 tune
variation~\cite{ATL-PHYS-PUB-2014-021} largely corresponds to the variation of
\alphas for ISR in the A14 tune.
The impact of final-state radiation (FSR) was evaluated by
varying the renormalisation scale for emissions from the
parton shower up and down by a factor of two.


The nominal \POWPY[8] sample was compared with an alternative
sample generated using the diagram subtraction
scheme~\cite{Frixione:2008yi,ATL-PHYS-PUB-2016-020} to estimate the
uncertainty arising from the interference with \ttbar production.

To evaluate the PDF uncertainties for the nominal PDF, the 100
variations for \NNPDF[3.0nlo] were taken into account.
In addition, the central value of this PDF was compared with the
central values of the \CT[14nnlo]~\cite{Dulat:2015mca} and
\MMHT[nnlo]~\cite{Harland-Lang:2014zoa} PDF sets.


\subsection[Powheg+Herwig7]{\POWHER[7]}
\label{subsubsec:tW_PH7}

\paragraph{Samples}
%\label{par:tW_PH7_samples}

\Cref{tab:tW_PH7} gives the DSIDs of the \(tW\) \POWHER[7] DR samples.
Single-top and single-anti-top (\(tW^-\) and \(\bar{t}W^+\)) events were generated in different samples.
The dileptonic samples overlap with the inclusive ones.

\begin{table}[htbp]
  \caption{Single-top \(tW\) associated production samples produced with \POWHER[7].}%
  \label{tab:tW_PH7}
  \centering
  \begin{tabular}{l l}
    \toprule
    DSID & Description \\
    \midrule
    411036 & \(tW^-\) (DR) inclusive \\
    411037 & \(\bar{t}W^+\) (DR) inclusive \\
    411038 & \(tW^-\) (DR) dileptonic \\
    411039 & \(\bar{t}W^+\) (DR) dileptonic \\
    \bottomrule
  \end{tabular}
\end{table}

\paragraph{Short description:}

The uncertainty due to the parton shower and hadronisation model was
evaluated by comparing the nominal sample of events with a sample where
events generated with the
\POWHEGBOX[v2]~\cite{Re:2010bp,Nason:2004rx,Frixione:2007vw,Alioli:2010xd}
generator were interfaced to
\HERWIG[7.04]~\cite{Bahr:2008pv,Bellm:2015jjp}, using the H7UE set
of tuned parameters~\cite{Bellm:2015jjp} and the \MMHT[lo] PDF set~\cite{Harland-Lang:2014zoa}.


\paragraph{Long description:}

The impact of using a different parton shower and hadronisation model was evaluated
by comparing the nominal \(tW\) sample with another sample produced with the
\POWHEGBOX[v2]~\cite{Re:2010bp,Nason:2004rx,Frixione:2007vw,Alioli:2010xd}
generator but interfaced with \HERWIG[7.04]~\cite{Bahr:2008pv,Bellm:2015jjp},
using the H7UE set of tuned parameters~\cite{Bellm:2015jjp} and the
\MMHT[lo] PDF set \cite{Harland-Lang:2014zoa}.
\POWHEGBOX provided matrix elements at next-to-leading order~(NLO) in the
strong coupling constant \alphas in the five-flavour scheme with the
\NNPDF[3.0nlo]~\cite{Ball:2014uwa} parton distribution function~(PDF).
The functional form of the renormalisation and factorisation scales was set to
the default scale, which is equal to the top-quark mass.  The diagram removal
scheme~\cite{Frixione:2008yi} was employed to handle the interference
with \ttbar production~\cite{ATL-PHYS-PUB-2016-020}.  The decays of bottom
and charm hadrons are simulated using the \EVTGEN[1.6.0]
program~\cite{Lange:2001uf}.


\subsection[MadGraph5\_aMC@NLO+Pythia8]{\MGNLOPY[8]}
\label{subsubsec:tW_aMCP8}

\paragraph{Samples}
%\label{par:tW_aMCP8_samples}

\Cref{tab:tW_aMCP8} gives the DSIDs of the \(tW\) \MGNLOPY[8] samples.
The dileptonic sample overlaps with the inclusive one.

\begin{table}[htbp]
  \caption{Single-top \(tW\) associated production samples produced with \MGNLOPY[8].}%
  \label{tab:tW_aMCP8}
  \centering
  \begin{tabular}{l l}
    \toprule
    DSID & Description \\
    \midrule
    412002 & \(tW\) inclusive \\
    412003 & \(tW\) dileptonic \\
    \bottomrule
  \end{tabular}
\end{table}

\paragraph{Short description:}

To assess the uncertainty in the matching of NLO matrix elements to the
parton shower, the nominal \(tW\) sample was compared with a sample generated
with the \MGNLO[2.6.2]~\cite{Alwall:2014hca} generator at NLO in QCD using the five-flavour
scheme and the \NNPDF[2.3nlo]~\cite{Ball:2014uwa} PDF set. The events were
interfaced with \PYTHIA[8.230]~\cite{Sjostrand:2014zea}, using the A14
set of tuned parameters~\cite{ATL-PHYS-PUB-2014-021} and the \NNPDF[2.3lo]
PDF.


\paragraph{Long description:}

To assess the uncertainty due to the choice of matching scheme, the nominal \(tW\) sample was compared with a sample generated
with the \MGNLO[2.6.2]~\cite{Alwall:2014hca} generator, which provided matrix elements at next-to-leading order~(NLO) in the strong coupling constant \alphas
in the five-flavour scheme, using the \NNPDF[2.3nlo]~\cite{Ball:2014uwa} PDF set.
The functional form of the renormalisation and factorisation scale was set to the default scale, which is equal to the top-quark mass.
The parton-shower starting scale had the functional form \(\muQ = \HT/2\)~\cite{ATL-PHYS-PUB-2017-007},
where \HT is defined as the scalar sum of the \pT of all outgoing partons.
The diagram removal scheme~\cite{Frixione:2008yi} was employed to handle the interference with \ttbar production~\cite{ATL-PHYS-PUB-2016-020}.
The events were interfaced with \PYTHIA[8.230]~\cite{Sjostrand:2014zea}, using the A14 set of tuned parameters~\cite{ATL-PHYS-PUB-2014-021}
and the \NNPDF[2.3lo] PDF.
The decays of bottom and charm hadrons were simulated using the \EVTGEN[1.6.0] program~\cite{Lange:2001uf}.


%%%%%%%%%%%%%%%%%%%%%%%%%%%%%%%%%%%%%%%%%%%
%%%              t-channel              %%%
%%%%%%%%%%%%%%%%%%%%%%%%%%%%%%%%%%%%%%%%%%%
\section{Single-top \texorpdfstring{\(t\)}{t}-channel production}
\label{subsec:tchan}

This section describes the MC samples used for the modelling of single-top \(t\)-channel production.
\Cref{subsubsec:tchan_PP8} describes the \POWPY[8] samples used for the nominal prediction
and for the uncertainty from additional radiation and due to PDFs.
\Cref{subsubsec:tchan_PH7} describes the \POWHER[7] samples used for the uncertainty due to the choice of parton shower and hadronisation model,
and \cref{subsubsec:tchan_aMCP8} describes the \MGNLOPY[8] samples used for the uncertainty due to the choice of matching scheme.

The reference cross-section values are extracted from Ref.~\cite{LHCTopWGsgtopXsec}.


\subsection[Powheg+Pythia8]{\POWPY[8]}
\label{subsubsec:tchan_PP8}

\paragraph{Samples}
%\label{par:tchan_PP8_samples}

\Cref{tab:tchan_PP8} gives the DSIDs of the \(t\)-channel \POWPY[8] samples.
Single-top and single-anti-top events were generated in distinct samples.

\begin{table}[!htbp]
  \caption{Single-top \(t\)-channel event samples produced with \POWPY[8].}%
  \label{tab:tchan_PP8}
  \centering
  \begin{tabular}{l l}
    \toprule
    DSID & Description \\
    \midrule
    410658 & \(t\)-channel \(t\) leptonic \\
    410659 & \(t\)-channel \(\bar{t}\) leptonic \\
    \bottomrule
  \end{tabular}
\end{table}

\paragraph{Short description:}

Single-top \(t\)-channel production was modelled using the
\POWHEGBOX[v2]~\cite{Frederix:2012dh,Nason:2004rx,Frixione:2007vw,Alioli:2010xd}
generator at NLO in QCD using the four-flavour scheme and the
corresponding \NNPDF[3.0nlo] set of PDFs~\cite{Ball:2014uwa}.  The events were
interfaced with \PYTHIA[8.230]~\cite{Sjostrand:2014zea} using the A14
tune~\cite{ATL-PHYS-PUB-2014-021} and the \NNPDF[2.3lo] set of
PDFs~\cite{Ball:2012cx}.

The uncertainty due to initial-state radiation (ISR) was estimated by
simultaneously varying the \hdamp parameter and the \muR and
\muF scales, and choosing the Var3c up/down variants of the A14 tune
as described in Ref.~\cite{ATL-PHYS-PUB-2017-007}. The impact of
final-state radiation (FSR) was evaluated by varying the renormalisation scale
for emissions from the parton shower up or down by a factor two.



\paragraph{Long description:}

Single-top \(t\)-channel production was modelled using the
\POWHEGBOX[v2]~\cite{Frederix:2012dh,Nason:2004rx,Frixione:2007vw,Alioli:2010xd}
generator, which provided matrix elements at next-to-leading-order~(NLO)
accuracy in the strong coupling constant \alphas\ in the four-flavour
scheme with the corresponding \NNPDF[3.0nlo]~\cite{Ball:2014uwa} parton
distribution function~(PDF) set.  The functional form of the
renormalisation and factorisation scales was set to
\(\sqrt{m_b^2 + \pTX[2][b]}\) following the
recommendation of Ref.~\cite{Frederix:2012dh}. Top quarks were decayed at
LO using \MADSPIN~\cite{Frixione:2007zp,Artoisenet:2012st} to preserve
all spin correlations.  The events were interfaced with
\PYTHIA[8.230]~\cite{Sjostrand:2014zea} using the A14
tune~\cite{ATL-PHYS-PUB-2014-021} and the \NNPDF[2.3lo] PDF set.
The decays of bottom and charm hadrons were simulated using the
\EVTGEN[1.6.0] program~\cite{Lange:2001uf}.

The inclusive cross-section was corrected to the theory prediction calculated at NLO in QCD with
\HATHOR[2.1]~\cite{Aliev:2010zk,Kant:2014oha}.
For proton--proton collisions at a centre-of-mass energy of \(\rts = \qty{13}{\TeV}\), this cross-section corresponds to
\(\sigma(t,t\text{-chan})_\text{NLO}= 136.02^{+5.40}_{-4.57}\)\,pb (\(\sigma(\bar{t},t\text{-chan})_\text{NLO}=80.95^{+4.06}_{-3.61}\)\,pb)
for single-top (single-anti-top) production, using a top-quark mass of \(\mtop = \qty{172.5}{\GeV}\).
The uncertainties in the cross-section due to the PDF and \alphas were calculated using the \PDFforLHC prescription~\cite{Butterworth:2015oua}
with the \MSTW[nlo] 68\% CL~\cite{Martin:2009iq,Martin:2009bu}, \CT[10nlo]~\cite{Lai:2010vv}
and \NNPDF[2.3nlo]~\cite{Ball:2012cx} PDF sets,
and were added in quadrature to the effect of the scale uncertainty.

The uncertainty due to initial-state radiation (ISR) was estimated by
comparing the nominal \ttbar sample with two additional
samples~\cite{ATL-PHYS-PUB-2017-007}.  To simulate higher parton
radiation, the factorisation and renormalisation scales were reduced by
a factor of 0.5 while simultaneously increasing the \hdamp value to
3.0\,\mtop and using the Var3c up variation from the A14 tune. For
lower parton radiation, \muR and \muF were increased by a factor of two
while keeping the \hdamp value set to 1.5\,\mtop and using the Var3c down
variation in the parton shower.  The Var3c A14 tune
variation~\cite{ATL-PHYS-PUB-2014-021} largely corresponds to the variation of
\alphas for ISR in the A14 tune.
The impact of final-state radiation (FSR) was evaluated by
varying the renormalisation scale for emissions from the
parton shower up and down by a factor of two.


To evaluate the PDF uncertainties for the nominal PDF, the 100 variations for \NNPDF[3.0nlo] were taken into account.
In addition, the central value of this PDF was compared with the central values of the \CT[14nnlo]~\cite{Dulat:2015mca}
and \MMHT[nnlo]~\cite{Harland-Lang:2014zoa} PDF sets.


\subsection[Powheg+Herwig7]{\POWHER[7]}
\label{subsubsec:tchan_PH7}

\paragraph{Samples}
%\label{par:tchan_PH7_samples}

\Cref{tab:tchan_PH7} gives the DSIDs of the \(t\)-channel \POWHER[7] samples.
Single-top and single-anti-top events were generated in distinct samples.

\begin{table}[htbp]
  \caption{Single-top \(t\)-channel event samples produced with \POWHER[7].}%
  \label{tab:tchan_PH7}
  \centering
  \begin{tabular}{l l}
    \toprule
    DSID & Description \\
    \midrule
    411032 & \(t\)-channel \(\bar t\) leptonic \\
    411033 & \(t\)-channel \(t\) leptonic \\
    \bottomrule
  \end{tabular}
\end{table}

\paragraph{Short description:}

The uncertainty due to the parton shower and hadronisation model was
evaluated by comparing the nominal sample of events with a sample where
the events generated with the
\POWHEGBOX[v2]~\cite{Frederix:2012dh,Nason:2004rx,Frixione:2007vw,Alioli:2010xd}
generator were interfaced to
\HERWIG[7.04]~\cite{Bahr:2008pv,Bellm:2015jjp}, using the H7UE set
of tuned parameters~\cite{Bellm:2015jjp} and the \MMHT[lo] PDF set
\cite{Harland-Lang:2014zoa}.


\paragraph{Long description:}

The impact of using a different parton shower and hadronisation model was evaluated by comparing the nominal sample
with another sample produced with the \POWHEGBOX[v2]~\cite{Frederix:2012dh,Nason:2004rx,Frixione:2007vw,Alioli:2010xd}
generator but interfaced with \HERWIG[7.04]~\cite{Bahr:2008pv,Bellm:2015jjp}, using the H7UE set of
tuned parameters~\cite{Bellm:2015jjp} and the \MMHT[lo] PDF set \cite{Harland-Lang:2014zoa}.
\POWHEGBOX provided matrix elements at next-to-leading order~(NLO) in the strong coupling constant \alphas
in the four-flavour scheme with the corresponding \NNPDF[3.0nlo]~\cite{Ball:2014uwa} parton distribution function~(PDF).
The functional form of the renormalisation and factorisation scales was set to \(\sqrt{m_b^2 + \pTX[2][b]}\)
following the recommendation of Ref.~\cite{Frederix:2012dh}.
Top quarks were decayed at LO using \MADSPIN~\cite{Frixione:2007zp,Artoisenet:2012st} to preserve all spin correlations.
The decays of bottom and charm hadrons were simulated using the \EVTGEN[1.6.0] program~\cite{Lange:2001uf}.


\subsection[MadGraph5\_aMC@NLO+Pythia8]{\MGNLOPY[8]}
\label{subsubsec:tchan_aMCP8}

\paragraph{Samples}
%\label{par:tchan_aMCP8_samples}

\Cref{tab:tchan_aMCP8} gives the DSIDs of the \(t\)-channel \MGNLOPY[8] samples.

\begin{table}[htbp]
  \caption{Single-top \(t\)-channel event samples produced with \MGNLOPY[8].}%
  \label{tab:tchan_aMCP8}
  \centering
  \begin{tabular}{l l}
    \toprule
    DSID & Description \\
    \midrule
    412004 & \(t\)-channel leptonic \\
    \bottomrule
  \end{tabular}
\end{table}

\paragraph{Short description:}

To assess the uncertainty in the matching of NLO matrix elements to the
parton shower, the nominal sample was compared with a sample generated
with the \MGNLO[2.6.2]~\cite{Alwall:2014hca} generator at NLO in QCD using the five-flavour
scheme and the \NNPDF[2.3nlo]~\cite{Ball:2014uwa} PDF set. The events were
interfaced with \PYTHIA[8.230]~\cite{Sjostrand:2014zea}, using the A14
set of tuned parameters~\cite{ATL-PHYS-PUB-2014-021} and the \NNPDF[2.3lo] PDF set.


\paragraph{Long description:}

To assess the uncertainty due to the choice of matching scheme, the nominal sample was compared with a sample generated
with the \MGNLO[2.6.2]~\cite{Alwall:2014hca} generator, which provided matrix elements at next-to-leading order~(NLO) in the strong coupling constant \alphas
in the four-flavour scheme, using the corresponding \NNPDF[3.0nlo]~\cite{Ball:2014uwa} PDF set.
The functional form of the renormalisation and factorisation scales was set to \(\sqrt{m_b^2 + \pTX[2][b]}\)
following the recommendation of Ref.~\cite{Frederix:2012dh}.
The parton-shower starting scale had the functional form \(\muQ = \HT/2\)~\cite{ATL-PHYS-PUB-2017-007},
where \HT is defined as the scalar sum of the \pT of all outgoing partons.
Top quarks were decayed at LO using \MADSPIN~\cite{Frixione:2007zp,Artoisenet:2012st} to preserve all spin correlations.
The events were interfaced with \PYTHIA[8.230]~\cite{Sjostrand:2014zea}, using the A14 set of tuned parameters~\cite{ATL-PHYS-PUB-2014-021}
and the \NNPDF[2.3lo] PDF set.
The decays of bottom and charm hadrons were simulated using the \EVTGEN[1.6.0] program~\cite{Lange:2001uf}.


%%%%%%%%%%%%%%%%%%%%%%%%%%%%%%%%%%%%%%%%%%%
%%%              s-channel              %%%
%%%%%%%%%%%%%%%%%%%%%%%%%%%%%%%%%%%%%%%%%%%
\subsection{Single-top \texorpdfstring{\(s\)}{s}-channel production}
\label{subsec:schan}

This section describes the MC samples used for the modelling of single-top \(s\)-channel production.
\Cref{subsubsec:schan_PP8} describes the \POWPY[8] samples used for the nominal prediction
and for the uncertainty from additional radiation and due to PDFs.
\Cref{subsubsec:schan_PH7} describes the \POWHER[7] samples used for the uncertainty due to the parton shower and hadronisation model,
and \cref{subsubsec:schan_aMCP8} describes the \MGNLOPY[8] samples used for the uncertainty due to the choice of matching scheme.

The reference cross-section values are extracted from Ref.~\cite{LHCTopWGsgtopXsec}.

\subsection[Powheg+Pythia8]{\POWPY[8]}
\label{subsubsec:schan_PP8}

\paragraph{Samples}
%\label{par:schan_PP8_samples}

\Cref{tab:schan_PP8} gives the DSIDs of the \(s\)-channel \POWPY[8] samples.
Single-top and single-anti-top events were generated in distinct samples.

\begin{table}[htbp]
  \caption{Single-top \(s\)-channel event samples produced with \POWPY[8].}%
  \label{tab:schan_PP8}
  \centering
  \begin{tabular}{l l}
    \toprule
    DSID & Description \\
    \midrule
    410644 & \(s\)-channel \(t\) leptonic \\
    410645 & \(s\)-channel \(\bar{t}\) leptonic \\
    \bottomrule
  \end{tabular}
\end{table}

\paragraph{Short description:}

Single-top \(s\)-channel production was modelled using the \POWHEGBOX[v2]~\cite{Alioli:2009je,Nason:2004rx,Frixione:2007vw,Alioli:2010xd}
generator at NLO in QCD in the five-flavour scheme with the \NNPDF[3.0nlo]~\cite{Ball:2014uwa} parton distribution function~(PDF) set.
The events were interfaced with \PYTHIA[8.230]~\cite{Sjostrand:2014zea} using the A14 tune~\cite{ATL-PHYS-PUB-2014-021} and the
\NNPDF[2.3lo] PDF set.

The uncertainty due to initial-state radiation (ISR) was estimated by
simultaneously varying the \hdamp parameter and the \muR and
\muF scales, and choosing the Var3c up/down variants of the A14 tune
as described in Ref.~\cite{ATL-PHYS-PUB-2017-007}. The impact of
final-state radiation (FSR) was evaluated by varying the renormalisation scale
for emissions from the parton shower up or down by a factor two.



\paragraph{Long description:}

Single-top \(s\)-channel production was modelled using the \POWHEGBOX[v2]~\cite{Alioli:2009je,Nason:2004rx,Frixione:2007vw,Alioli:2010xd}
generator, which provided matrix elements at next-to-leading order~(NLO) in the strong coupling constant \alphas in the
five-flavour scheme with the \NNPDF[3.0nlo]~\cite{Ball:2014uwa} parton distribution function~(PDF) set.
The functional form of the renormalisation and factorisation scales was set to the default scale, which was equal to the top-quark mass.
The events were interfaced with \PYTHIA[8.230]~\cite{Sjostrand:2014zea} using the A14 tune~\cite{ATL-PHYS-PUB-2014-021} and the
\NNPDF[2.3lo] PDF set.
The decays of bottom and charm hadrons were simulated using the \EVTGEN[1.6.0] program~\cite{Lange:2001uf}.

The inclusive cross-section was corrected to the theory prediction calculated at NLO in QCD with
\HATHOR[2.1]~\cite{Aliev:2010zk,Kant:2014oha}.
For proton--proton collisions at a centre-of-mass energy of \(\rts = \qty{13}{\TeV}\), this cross-section corresponds to
\(\sigma(t,s\text{-chan})_{\text{NLO}} = 6.35^{+0.23}_{-0.20}\,\unit{\pb}\)
(\(\sigma(\bar{t},s\text{-chan})_{\text{NLO}} = 3.97^{+0.19}_{-0.17}\,\unit{\pb}\))
for single-top (single-anti-top) production, using a top-quark mass of \(\mtop = \qty{172.5}{\GeV}\).
The uncertainties in the cross-section due to the PDF and \alphas were calculated using the \PDFforLHC prescription~\cite{Butterworth:2015oua}
with the \MSTW[nlo] 68\% CL~\cite{Martin:2009iq,Martin:2009bu}, \CT[10nlo]~\cite{Lai:2010vv} and \NNPDF[2.3nlo]~\cite{Ball:2012cx} PDF sets,
and were added in quadrature to the effect of the scale uncertainty.

The uncertainty due to initial-state radiation (ISR) was estimated by
comparing the nominal \ttbar sample with two additional
samples~\cite{ATL-PHYS-PUB-2017-007}.  To simulate higher parton
radiation, the factorisation and renormalisation scales were reduced by
a factor of 0.5 while simultaneously increasing the \hdamp value to
3.0\,\mtop and using the Var3c up variation from the A14 tune. For
lower parton radiation, \muR and \muF were increased by a factor of two
while keeping the \hdamp value set to 1.5\,\mtop and using the Var3c down
variation in the parton shower.  The Var3c A14 tune
variation~\cite{ATL-PHYS-PUB-2014-021} largely corresponds to the variation of
\alphas for ISR in the A14 tune.
The impact of final-state radiation (FSR) was evaluated by
varying the renormalisation scale for emissions from the
parton shower up and down by a factor of two.


To evaluate the PDF uncertainties for the nominal PDF, the 100 variations for \NNPDF[3.0nlo] were taken into account.
In addition, the central value of this PDF was compared with the central values of the
\CT[14nnlo]~\cite{Dulat:2015mca} and \MMHT[nnlo]~\cite{Harland-Lang:2014zoa} PDF sets.


\subsection[Powheg+Herwig7]{\POWHER[7]}
\label{subsubsec:schan_PH7}

\paragraph{Samples}
%\label{par:schan_PH7_samples}

\Cref{tab:schan_PH7} gives the DSIDs of the \(s\)-channel \POWHER[7] samples.
Single-top and single-anti-top events were generated in distinct samples.

\begin{table}[htbp]
  \caption{Single-top \(s\)-channel event samples produced with \POWHER[7].}%
  \label{tab:schan_PH7}
  \centering
  \begin{tabular}{l l}
    \toprule
    DSID & Description \\
    \midrule
    411034 & \(s\)-channel \(t\) leptonic \\
    411035 & \(s\)-channel \(\bar{t}\) leptonic \\
    \bottomrule
  \end{tabular}
\end{table}

\paragraph{Short description:}

The impact of using a different parton shower and hadronisation model was evaluated by comparing the nominal sample
with another sample produced with the \POWHEGBOX[v2]~\cite{Alioli:2009je,Nason:2004rx,Frixione:2007vw,Alioli:2010xd}
generator at NLO in the strong coupling constant \alphas in the five-flavour scheme using the
\NNPDF[3.0nlo]~\cite{Ball:2014uwa} parton distribution function~(PDF).
Events in the latter sample were interfaced with \HERWIG[7.04]~\cite{Bahr:2008pv,Bellm:2015jjp}, using the H7UE set of
tuned parameters~\cite{Bellm:2015jjp} and the \MMHT[lo] PDF set \cite{Harland-Lang:2014zoa}.


\paragraph{Long description:}

The impact of using a different parton shower and hadronisation model was evaluated by comparing the nominal sample
with another sample produced with the \POWHEGBOX[v2]~\cite{Alioli:2009je,Nason:2004rx,Frixione:2007vw,Alioli:2010xd}
generator but interfaced with \HERWIG[7.04]~\cite{Bahr:2008pv,Bellm:2015jjp}, using the H7UE set of
tuned parameters~\cite{Bellm:2015jjp} and the \MMHT[lo] PDF set \cite{Harland-Lang:2014zoa}.
\POWHEGBOX provided matrix elements at next-to-leading order~(NLO) in the strong coupling constant \alphas
in the five-flavour scheme with the \NNPDF[3.0nlo]~\cite{Ball:2014uwa} parton distribution function~(PDF).
The functional form of the renormalisation and factorisation scales was set to the default scale, which is equal to the top-quark mass.
The decays of bottom and charm hadrons were simulated using the \EVTGEN[1.6.0] program~\cite{Lange:2001uf}.


\subsection[MadGraph5\_aMC@NLO+Pythia8]{\MGNLOPY[8]}
\label{subsubsec:schan_aMCP8}

\paragraph{Samples}
%\label{par:schan_aMCP8_samples}

\Cref{tab:schan_aMCP8} gives the DSIDs of the \(s\)-channel \MGNLOPY[8] samples.

\begin{table}[htbp]
  \caption{Single-top \(s\)-channel event samples produced with \MGNLOPY[8].}%
  \label{tab:schan_aMCP8}
  \centering
  \begin{tabular}{l l}
    \toprule
    DSID & Description \\
    \midrule
    412005 & \(s\)-channel leptonic \\
    \bottomrule
  \end{tabular}
\end{table}

\paragraph{Short description:}

%% the MG5_aMC@NLO version was double checked in the tag collector
To assess the uncertainty due to the choice of matching scheme, the nominal sample was compared with a sample generated
with the \MGNLO[2.6.2]~\cite{Alwall:2014hca} generator at NLO in the strong coupling constant \alphas in the five-flavour scheme,
using the \NNPDF[3.0nlo]~\cite{Ball:2014uwa} PDF set.
The events were interfaced with \PYTHIA[8.230]~\cite{Sjostrand:2014zea}, using the A14 set of tuned parameters~\cite{ATL-PHYS-PUB-2014-021}
and the \NNPDF[2.3lo] PDF set.


\paragraph{Long description:}

To assess the uncertainty due to the choice of matching scheme, the nominal sample was compared with a sample generated
with the \MGNLO[2.6.2]~\cite{Alwall:2014hca} generator, which provided matrix elements at next-to-leading order~(NLO) in the strong coupling constant \alphas
in the five-flavour scheme with the \NNPDF[3.0nlo]~\cite{Ball:2014uwa} parton distribution function~(PDF).
The functional form of the renormalisation and factorisation scales was set to the default scale, which is equal to the top-quark mass.
The parton-shower starting scale had the functional form \(\muQ = \HT/2\)~\cite{ATL-PHYS-PUB-2017-007},
where \HT is defined as the scalar sum of the \pT of all outgoing partons.
Top quarks were decayed at LO using \MADSPIN~\cite{Frixione:2007zp,Artoisenet:2012st} to preserve all spin correlations.
The events were interfaced with \PYTHIA[8.230]~\cite{Sjostrand:2014zea}, using the A14 set of tuned parameters~\cite{ATL-PHYS-PUB-2014-021}
and the \NNPDF[2.3lo] PDF set.
The decays of bottom and charm hadrons were simulated using the \EVTGEN[1.6.0] program~\cite{Lange:2001uf}.
